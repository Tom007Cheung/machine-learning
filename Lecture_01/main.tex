\documentclass[a4paper, 12pt]{article}
\usepackage[utf8]{inputenc}

\usepackage{xcolor}
\usepackage{enumitem}
\usepackage{mathtools}
\usepackage{todonotes}
\usepackage{tikz,lipsum,lmodern}
\usepackage[most]{tcolorbox}
\usepackage{hyperref}
\usepackage{pifont}
\hypersetup{
    % bookmarks=true,         % show bookmarks bar?
    unicode=false,          % non-Latin characters in Acrobat’s bookmarks
    pdftoolbar=true,        % show Acrobat’s toolbar?
    pdfmenubar=true,        % show Acrobat’s menu?
    pdffitwindow=false,     % window fit to page when opened
    pdfstartview={FitH},    % fits the width of the page to the window
    pdftitle={Lecture 01},    % title
    pdfauthor={Tom Cheung},     % author
    pdfsubject={machine learning},   % subject of the document
    pdfcreator={overleaf},   % creator of the document
    pdfproducer={https://www.overleaf.com/}, % producer of the document
    pdfkeywords={definition, supervised learning, linear regression}, % list of keywords
    pdfnewwindow=true,      % links in new PDF window
    colorlinks=true,       % false: boxed links; true: colored links
    linkcolor=red,          % color of internal links (change box color with linkbordercolor)
    citecolor=green,        % color of links to bibliography
    filecolor=magenta,      % color of file links
    urlcolor=orange          % color of external links
}

\title{Lecture 01}
\author{Tom Cheung}
\date{October 2018}

\begin{document}

\maketitle
\part*{Machine Learning}
\section{Definition}
    \begin{enumerate}
        \item \href{https://ieeexplore.ieee.org/document/5392560}{Arthur Samuel}
        \item \href{http://www.cs.cmu.edu/~tom/}{Tom M. Mitchell}
    \end{enumerate}
 
\section{Parts}
    \begin{itemize}
        \item supervised learning
            \begin{itemize}
                \item Regression
                \item Classification
            \end{itemize}
        \item learning theory 
        \item unsupervised learning
            \begin{itemize}
                \item \textbf{ICA algorithm} \todo{stay tuned}
            \end{itemize}
        \item reinforcement learning
    \end{itemize}
\part*{Supervised learning}
\begin{center}
    \begin{tabular}{ l | r }		
        \textbf{feature} & \textbf{target} \\\hline
        ~~~$x^{(0)}$ & $y^{(0)}$ \\ 
        ~~~$x^{(1)}$ & $y^{(1)}$ \\
        ~~~\vdots    & \vdots \\ 
        ~~~$x^{(i)}$ & $y^{(i)}$
    \end{tabular}
\end{center}
\begin{itemize}
    \item \textcolor{cyan}{training example: }    $(x^{(i)},~y^{(i)})$
    \item \textcolor{cyan}{training set: } ~~~~~~~$\left\{(x^{(i)},~y^{(i)});i = 1, \cdots, m\right\}$
    \item space of input and output               \textcolor{red}{$\mathcal{X}, \mathcal{Y}$}
    \item hypothesis: ~~~~~~~~~~                  $h: $ $\mathcal{X} \mapsto \mathcal{Y}$
\end{itemize}\par\par
\begin{description}[align=left]
\item [\textbf{REGRESSION}] ~~~~~~~\textit{\textcolor{blue}{continuous}}
\item [\textbf{CLASSIFICATION}]    \textit{\textcolor{blue}{discrete}}
\end{description}
\section{Linear Regression}
\framebox{feature selection\textcolor{red}{\textit{\textbf{!?}}}} \todo{stay tuned} \\

% https://www.overleaf.com/latex/examples/drawing-coloured-boxes-using-tcolorbox/pvknncpjyfbp

\begin{tcolorbox}[colback=red!5!white,colframe=red!75!black]
    \begin{equation}
        h_\theta(x) = \theta_{0} + \theta_{1}x_{1} + \theta_{2}x_{2}
    \end{equation}
   \tcblower
  $\theta_{i}: $ parameter or weight ~~~~~~~~~~~~~~~~~~~~~~~~~~~ $x_{0} = 1: $ intercept term
\end{tcolorbox}

\begin{tcolorbox}[colback=red!5!white,colframe=red!75!black]
    \begin{equation}
        h(x) = \sum_{i = 0}^{n} \theta_{i}x_{i} = \theta^{\mathrm{T}}x
    \end{equation}
\end{tcolorbox}

\begin{tcolorbox}[enhanced,attach boxed title to top center={yshift=-3mm,yshifttext=-1mm},
  colback=blue!5!white,colframe=blue!75!black,colbacktitle=red!80!black,
  title=cost function,fonttitle=\bfseries,
  boxed title style={size=small,colframe=red!50!black} ]
    \begin{equation}
        J(\theta) = \frac{1}{2}\sum_{i = 1}^{m}(h_{\theta}(x^{(i)}) - y^{(i)})^2
    \end{equation}
\end{tcolorbox}
\framebox{{\textit{The ordinary least squares
regression model}~\textcolor{red}{\textbf{\textit{!?}}}}}
\footnote{\href{https://en.wikipedia.org/wiki/Overdetermined_system}{The method of ordinary least squares can be used to find an approximate solution to overdetermined systems.} \\
\textit{see}~\ding{43}: \textbf{\textcolor{blue}{Approximate solutions}}}

\end{document}
